\documentclass{article}
\usepackage{graphicx,fancyhdr,amsmath,amssymb,amsthm,subfig,url,hyperref}
\usepackage[margin=1in]{geometry}
\usepackage{xltxtra}
\usepackage{xgreek}
\usepackage{amsfonts}
\usepackage{amssymb}
\usepackage{amsmath}
\usepackage{amsthm}
\usepackage{mathtools}

\usepackage{caption}
\usepackage{subfig}
\setmainfont[Mapping=tex-text]{Times New Roman}
%----------------------- Macros and Definitions --------------------------

%% FILL THIS OUT
\newcommand{\studentname}{Νικόλαος Ζαρίφης}
\newcommand{\suid}{03112178}
\newcommand{\exerciseset}{Exercise Set 3}
%% END



\renewcommand{\theenumi}{\bf \Alph{enumi}}

%\theoremstyle{plain}
%\newtheorem{theorem}{Theorem}
%\newtheorem{lemma}[theorem]{Lemma}

\fancypagestyle{plain}{}
\pagestyle{fancy}
\fancyhf{}
\fancyhead[RO,LE]{\bfseries\large NTUAthens}
\fancyhead[LO,RE]{\bfseries\large Θεωρία γράφων}
\fancyfoot[LO,RE]{\bfseries\large \studentname: nick.zarifis@hotmail.com}
\fancyfoot[RO,LE]{\bfseries\thepage}
\renewcommand{\headrulewidth}{1pt}
\renewcommand{\footrulewidth}{1pt}

\graphicspath{{figures/}}
\usepackage{tikz}
%-------------------------------- Title ----------------------------------

\title{Θεωρία γράφων \\ \exerciseset}
\author{\studentname \qquad  ID: \suid}

%--------------------------------- Text ----------------------------------
\DeclarePairedDelimiter\floor{\lfloor}{\rfloor}
\newtheorem{lemma}{Lemma}
\begin{document}
\maketitle
\section*{Άσκηση 1}
1)Έχουμε 2E = $\sum_{u\in E}d(u) > \frac{k-2}{k-1}n^2 $ , από γνωστό λήμμα: x(G)> $\frac{n^2}{n^2-2E} > k-1$ από παραπάνω σχέση, άρα το γράφημα δεν μπορεί να είναι k-1 χροματίσημο, άρα χ(G)=k, αφού είναι κ χρωματίσημο. 

2)$\rightarrow$Αν ισχυεί αυτό τότε είναι 2μερες το γράφημα , οπότε οι 2 διαφορετικές κορυφές βρίσκονται σε διαφορετικό σύνολο, όμως αφού απο μια 3η έχουν ίδια απόσταση , αφού το 2-μερές θέλει ζύγα βήματα για να βρεθεί στο ίδιο σύνολο άρα στο ίδιο χρώμα είτε στο τέλος θα βρίσκεται η 3η κορυφή στο ίδιο σύνολο που ξεκήνησε το μονοπάτι είτε στο άλλο κι για της 2 κορυφές, όμως αυτό είναι άτοπο γιατί η κορυφή βγαίνει κισ τις 2 περιπτώσεις ότι ανήκει σε 2 σύνολα , πράγμα που δεν ισχυεί σε δημερές.
$\leftarrow$ Αντίστροφα αν έχουν τις ίδιες αποστάσεις τότε έχουμε αυτές οι 3ης αποστάσεις δημιουργούν κύκλο περιτού μήκους, όμως ξέρουμε οτι οι δημεροίς γράφοι έχουν άρτιο μήκος κύκλων άρα δεν είναι δημερές, όποτε έχει χροματικό αριθμό μεγαλήτερο του 2.

\section*{Άσκηση 2}
Έστω λοιπον ο $C_{2k+1}$, πέρνοντας το μέγιστο ταίριασμα του. Αφτό είανι ένα σύνολο απο κ ακμές (Παίρνουμε μια, αφήνουμε μια κτλπ)κι είναι το μέγιστο ταιριασμα του.Απο εδώ μπορούμε να αφαίρουμε μια ακμή κι να προσθέτουμε άλλη ωστέ οι κορυφές που δείχνουν κάθε ταίριασμα να ενόνονται. Αν λοιπόν αφαίρεσουμε την ($v_{2i-3}v_{2i}$) και βάλουμε την $v_{2i-4}v_{2i-3}$ θα έχουμε ένα matching graph που είναι το ίδοο με το $C_{2k+1}$
\section*{Άσκηση 3}
Σε ένα γράφημα το μέγιστο πλήθος ανεξάρτητων ακμών είναι το μέγιστο πλήθως που μπορεί να έχει ενα χρώμα ώστε να γίνεται νόμιμος ο χρωματισμός του. Σε ένα απλό γραφημά έχουμε $x(G)=\Delta(G)$ ή +1 , όποτε, αν γίνοταν με μόνο το μέγιστο βαθμό θα είχαμε στην χειρότερη πληθως ανεξάρτητων * $\Delta(G)$. Όμως εδώ έχουμε παραπάνω ακμές άρα απο περιστερώνα έχουμε $\Delta(G)+1$.

Για το δεύτερο ερώτημα, έχουμε $\floor{n/2} >a(G)$ (γιατί κάθε ακμή συνδέεται με 2 κορυφές) άρα η ανίσωστηα πας πάει στο πρώτο ερώτημα πάλι.
\section*{Άσκηση 4}
1) Αφού έιναι δέντρο όποιεδημοποτε κ κορυφές έχουν μεταξύ τους το πολύ κ-1 ακμές, πέρνωντας οποιεςσήποτε 5 κορυφές κι προσθέτωντας 3 ακμές , βλέπουμε ότι έχουμε 7 ακμές το πολύ, όμως το $K_5$ έχει 10 ακμές άρα σίγουρα δεν είανι ομομορφικό με το $ Κ_5$, (Δεν χρειάζεται να προσθέσουμε κουφές γιατί όταν τις αφαιρέσουμε πάμε στο ίδιο αποτέλεσμα λόγο δέντρου),Όμοιος δεν είναι με το $K_{3,3}$ γιατί 6 κορυφές θα είχαν 5 +3 = 8 , κι αυτό έχει 9 ακμές. Άρα είναι επίπεδο.
\\2) από το παραπάνω ερώτημα βλέπουμε ότι προσθέτωντας 2 ακμές ακόμα πάλι δεν γίνεται ομοιομορφικό με το $K_5$άρα πρέπει να δούμε μόνο για το $Κ_{3,3}$.Αλλά απο εκφώνηση βλέπουμε ότι δεν το περιέχει άρα είανι επίπεδο.
\section*{Άσκηση 5}
1) Για n>7 βλέπουμε ότι μπροούμε να πάρουμε 6 κορυφές οι όποιες είναι 2 $P_2$, βλέπουμε ότι το συμπλήρωμα τους θα είναι το $K_{3,3}$ με κάποιες ακμές παραπάνω, άρα δεν είναι επίπεδο. Αν είχαμε μια λιγότερη κορυφή τότε δεν μπορούμε να σχηματίσουμε το $K_{3,3}$,γιατί τα 2 μοναπάτια θα έχουν μια κοινή ακμή άρα το συμπλήρωμα δεν θα περιέχει το 3-μερές-πλήρες. για να έχουμε $K_5$ θα πρέπει να μπροούμε να βρούμε 5 ανεξάρτητες κορυφές, σε έναν κύκλο με n>10 μόνο , άρα για n<8 έχουμε επίπεδα μόνο.
Πιο απλή λύση :Χρησιμοποιώντας το λήμμα ότι αν είναι επίπεδο τότε m<3n-6 , έχουμε ότι το συμπληρωμάτικο θα έχει  m = $\frac{n*(n-1)}{2} - n$ ακμές , βάζοντας το στην ανισότητα κι λύνοντας το , βλέπουμε ότι ισχυεί για n<7.yy άρα για βρίσκουμε το ίδιο αποτέλεσμα με παραπάνω. Βέβαια εδω πρέει να δείξουμε με σχήμα οτι υπάρχει για n=5,6,7 .(για μικρότερα το ξέρουμε απο παλίες σειρες ασκήσεων).
\\2)οι απάντηση εδώ είναι τετριμένη, για κάθε ν είναι επίπεδος γιατί οι ακμές τους σχηματίζουν, έναν κύκλο κι 2 εσωτερικούς, που ο ενας εσωτερικός μπορεί να γραφτει εξωτερικά, άρα δεν τεμνονται κι είναι επίπεδο. 
\section*{Άσκηση 6}
1) Εύκολα, πέρνουμε μια κορυφή κι την ενόνουμε με όσες κορυφές είναι το Κ, σαν αστέρι. \\
2) για κ=2,3,4 έχουμε τα πλήρες γραφήματα . για κ=5 όμως: παίρνουμε το $K_4$ κι βάζουμε 1 κορυφή όπου την ενώνουμε με 2 κορύφες όπου έχουν μια ακμή με διαφορετικό χρώμα. κι έτσι βγάλαμε το αποτέλεσμα μας.
\end{document}
