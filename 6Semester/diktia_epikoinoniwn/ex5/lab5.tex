\documentclass{article}
\usepackage{graphicx,fancyhdr,amsmath,amssymb,amsthm,subfig,url,hyperref}
\usepackage[margin=1in]{geometry}
\usepackage{xltxtra}
\usepackage{xgreek}
\usepackage{amsfonts}
\usepackage{amssymb}
\usepackage{amsmath}
\usepackage{graphicx}
\usepackage{listings}
\usepackage{framed}
\usepackage{minted}
\usepackage{caption}
\usepackage{ subfig}
%\setmainfont[Mapping=tex-text]{Times New Roman}
\setmainfont{GFS Artemisia}
%----------------------- Macros and Definitions --------------------------
% FILL THIS OUT
\newcommand{\studentname}{Νικόλαος Ζαρίφης}
\newcommand{\suid}{03112178}
\newcommand{\exerciseset}{Πέμπτη εργαστηριακή Άσκηση}
% END



\renewcommand{\theenumi}{\bf \Alph{enumi}}

%\theoremstyle{plain}
%\newtheorem{theorem}{Theorem}
%\newtheorem{lemma}[theorem]{Lemma}

\fancypagestyle{plain}{}
\pagestyle{fancy}
\fancyhf{}
\fancyhead[RO,LE]{\bfseries\large NTUAthens}
\fancyhead[LO,RE]{\bfseries\large Δίκτυα επικοινωνιών}
\fancyfoot[LO,RE]{\bfseries\large \studentname: nick.zarifis@hotmail.com}
\fancyfoot[RO,LE]{\bfseries\thepage}
\renewcommand{\headrulewidth}{1pt}
\renewcommand{\footrulewidth}{1pt}

\graphicspath{{figures/}}

%-------------------------------- Title ----------------------------------

\title{Δίκτυα επικοινωνιών \\ \exerciseset}
\author{\studentname \qquad  ID: \suid}

%--------------------------------- Text ----------------------------------

\begin{document}
\maketitle
\begin{itemize}
	\item \textbf{Πώς πρέπει να τροποποιηθεί ο κώδικας της προσομοίωσης ώστε η ζεύξη μεταξύ των δύο κόμβων
		της διάταξης να απεικονίζεται σε οριζόντια θέση, όπως φαίνεται στο Σχήμα 1; } \\ Όπως γνωρίζουμε από τις προηγούμενες εργαστηριακές ασκήσεις αυτό γίνεται εύκολα με την εντολή:
	\$ns duplex-link-op $n(0) $n(1) orient right.\\ \\
	Το awk επεστεψε το ακόλουθο αποτέλεσμα:
	Total Data received	: 598960 Bytes\\
	Total Packets received	: 599
	\item \textbf{Να επαληθεύσετε κατά πόσον ισχύει ή όχι η εξίσωση σε περίπτωση απουσίας σφαλμάτων}
	Έχουμε:
	$W=10$, $TRANSP=\frac{960*8}{2Mb}=3,84*10^{-3}$,$PROP=40ms$\footnote{ όσο και η καθυσερηση ζευξης}.$TRANSA = \frac{40*8}{2Mb}=0.16*10^{-3}$
	$S=TRANSA+TRANSP +2*PROP=84.0ms$ άρα $\frac{10*3,84ms}{84.0ms}=0,45714$
	Βάζοντας το ακόλουθο κώδικα στο awk βλέπουμε πότε έφτασε το τελευταίο πακέτο: \inputminted[firstline=10,lastline=11]{awk}{lab5.awk}  (κι την τύπωση)
	Έχουμε το αποτέλεσμα: 5.33144s άρα, ρυθμός:$\frac{bytes*8}{time}=\frac{598960*8}{5.33144-0.25}=942976,78bits/s$ Άρα: η=$\frac{942976,78}{2Mbs}=0.471488$ Πράγματι είναι πολύ κοντά στην θεωρητική τιμή αλλά χάνει γιατί δεν έχουμε υπολογίσει τις επικεφαλίδες.
	\item \textbf{Ποιος είναι ο αριθμός των πακέτων που παρελήφθησαν; Πόσα δεδομένα παρελήφθησαν από τον
		παραλήπτη κατά τη διάρκεια της προσομοίωσης; } \\Η έξοδος του awk παραπάνω απαντά το ερώτημα.
		\item \textbf{ Τροποποιήστε κατάλληλα το πρόγραμμα awk, ώστε να προσδιορίζει τη συνολική διάρκεια
			μετάδοσης των δεδομένων (η οποία περιλαμβάνει και την ολοκλήρωση μετάδοσης όλων των
			επιβεβαιώσεων). Υπολογίστε το ρυθμό μετάδοσης δεδομένων και τη χρησιμοποίηση του καναλιού. } Στο β ερώτημα υπολόγισα τα παραπάνω . 
		Η χρησιμοποίηση του καναλιού είναι ίση με την μεταβλητή \textbf{η} του β ερωτήματος.
		\item \textbf{Με βάση την εξίσωση που παρατίθεται νωρίτερα, υπολογίστε τη θεωρητική τιμή της
			χρησιμοποίησης του καναλιού, θεωρώντας ότι το μέγεθος των πακέτων αυξάνεται κατά 40 byte
			λόγω επικεφαλίδων TCP και IP, και ότι οι επαληθεύσεις (ACK) έχουν μέγεθος 40 byte. Ισχύει η
			εξίσωση; Αν όχι, πού οφείλεται η απόκλιση; }. Ξαναυπολογίζοντας έχουμε: $W=10$, $TRANSP=\frac{1000*8}{2Mb}=4*10^{-3}$,$PROP=40ms$.$TRANSA = \frac{40*8}{2Mb}=0.16*10^{-3}$
		$S=TRANSA+TRANSP +2*PROP=84.16ms$ άρα $\frac{10*4ms}{84.16ms}=0,47529$.Η θεωρητική τιμή είναι μεγαλύτερη από την αυτη.Αυτό συμβαίνει εξαιτίας τις καθυστερήσεις μεταξύ λήψεις πακέτου κι αποστολείς επιβεβαιώσεις.
		\item \textbf{Διατηρώντας σταθερό το μέγεθος του παραθύρου, αλλάξτε το μήκος των πακέτων, ώστε η
			θεωρητική απόδοση του πρωτοκόλλου να λάβει τη μέγιστη τιμή της. Για ποιο μήκος πακέτων
			συμβαίνει αυτό; Υπολογίστε πειραματικά την απόδοση του πρωτοκόλλου (χρησιμοποίηση του
			καναλιού) για το μήκος πακέτου που προσδιορίσατε εδώ. Υπάρχει απόκλιση μεταξύ πειραματικής
			και θεωρητικής τιμής; }
		Θέτοντας την εξίσωση ίση με την μονάδα κι λύνοντας ως προς το μήκος έχομουμε ότι ισούτε με : 2231Bytes Τρέχοντας το script παίρνουμε:\\
		Total Data received	: 1251281 Bytes\\
		Total Packets received	: 551\\
		Last ack receive	: 5.335284\\
		Έτσι λοίπον : $\frac{1251281*8}{5.335284-0.25}=1968473,737160009$ .Άρα η απόδοση είναι: $\frac{1968473,737160009}{2Mbs}=0.98423=98\%$ .Βλέπουμε ότι είναι πολύ κοντά στην θεωρητική τιμή.
		\item \textbf{Διατηρώντας το μήκος πακέτου που υπολογίσατε στο ερώτημα (στ), αυξήστε (AM
			+6)=14 φορές το
			ρυθμό μετάδοσης της ζεύξης και ρυθμίστε το μέγεθος του παραθύρου, ώστε και πάλι η απόδοση να
			λάβει τη μέγιστη τιμή της. Για ποιο μέγεθος παραθύρου συμβαίνει αυτό; Πόσα περισσότερα bits
			απαιτούνται για την αναπαράσταση των αριθμών ακολουθίας πακέτων του πρωτοκόλλου Selective
			Repeat στην περίπτωση αυτή}\\Άρα θέτουμε τον ρυθμό μετάδοσης ίσο με 28Mb υπολογίζοντας πάλι το μήκος παραθύρου θα έχουμε:
			$W=\frac{S}{TRANSP}$. Κι έχουμε: $TRANSP = \frac{2231*8}{28Mb}=0.637$ και $TRANSA=\frac{40*8}{28Mb}=0.011ms$ $\rightarrow S=0.011 +0.637 +2*40=80.648ms$ περίπου.Άρα $W=126,67=127$.Τα bits απο θεωρία έχουμε ότι: $W=\frac{Max\_bits +1}{2}$Άρα $Max\_bits=2*W-1$. Αρχικά είχαμε $W=10$ άρα: $Max\_bits=log(19)=5$(πέρνουμε το ανώ ακέραιο) ενώ μετά: $Max\_bits=log(127*2-1)=log153=8$ δηλαδή 3 παραπάνω bits.
			\item \textbf{Εφαρμόστε τώρα το πρωτόκολλο για την παραμετροποίηση του ερωτήματος (ζ), θεωρώντας όμως
				ζεύξη με πενταπλάσια καθυστέρηση διάδοσης. Υπολογίστε την απόδοση του πρωτοκόλλου στη νέα
				αυτή ζεύξη τόσο θεωρητικά, όσο και πειραματικά. Αιτιολογείστε τυχόν αποκλίσεις που
				παρατηρούνται. }\\ 
			Αλλάζοντας τον κόδικα στα εξής σημεία:\\
			\textit{\$ns duplex-link \$n(0) \$n(1) 28Mb 200ms DropTail\\
				\$tcp0 set window\_ 127\\
			\$tcp0 set windowInit\_ 127}\\
		Κι τρέχοντας το awk έχουμε:\\
		Total Data received	: 3749381 Bytes\\
			Total Packets received	: 1651\\
			Last ack receive	: 5.540328\\
			
			
			Θεωρητικά η απόδοση είναι(χρησιμοποιόντας τους ίδιους τύπου πάλι):(αλλάζει μόνο το S=400,648)η=$\frac{W*TRANSP}{S}=293*0.274/(S-80+400)=0,200=20\%$ Πειραματικά έχουμε: ${3749381*8}{5.540328-0.25}=5669789,850459178 $ , αρα η$=\frac{2064287,033233581}{28Mb}=0.202492=20 \%$ Πολύ κοντά στην τυπμή
			\item \textbf{Εφαρμόστε το πρωτόκολλο Go Back N αντί του Selective Repeat στην τελευταία παραμετροποίηση
				της προσομοίωσης και μετρήστε την απόδοση του πρωτοκόλλου αυτού πειραματικά. Διαφέρουν οι
				πειραματικές αποδόσεις των δύο πρωτοκόλλων; Γιατί; }\\
			Αλλάζοντας την μεταβλητή tcp0 σε:\textit{ set tcp0 [new Agent/TCP/Reno]} κι τρέχοντας το awk έχουμε:\\
			Total Data received	: 3749381 Bytes\\
			Total Packets received	: 1651\\
			Last ack receive	: 5.540328\\
			Μετρώντας πειραματικά έχουμε: ρυθμό: $\frac{3749381*8}{5.540328-0.25}=5669789,850459178 $ και απόδοση: $\frac{5669789,850459178}{28Mb}=0.202492=20 \%$
			Βλέπουμε ότι το awk έβγαλε ακριβώς τα ίδια αποτελέσματα.Αυτό συμβαίνει γιατί έχουμε μια ιδανική προσομοίωσής. Αλλά σε πραγματικές καταστάσεις θα ήταν καλυτερο το selective γιατι θα έστελνε μόνο τα σφαλμένα πακέτα κι όχι όλο το παραθυρο.
		
		
	\end{itemize}
	
\end{document}
